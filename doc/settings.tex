\chapter{Settings}
\renewcommand{\arraystretch}{1}

Please refer to Figure \ref{fig:settings} for a visual representation of the settings.  A description of all settings is given in Table \ref{tab:settings}.  A description of all button functions is provided in Table \ref{tab:settings_buttons}.

% ------------------------------------------------
%
% Button Descriptions
%
% ------------------------------------------------

\newcommand{\buttontable}
	{
		\begin{table}[h!]
			\centering
			\begin{tabular}[t]{| p{2in} | p{2.1in} |}
			
				% Header
				\hline \textbf{Button:} & \textbf{Function:} \\ \hline \hline
				
				% Questions
				Load Settings & Load settings from a \texttt{.yaml} file.  This loads both basic and advanced settings \\ \hline
				Save Settings & Save settings to a \texttt{.yaml} file.  This saves both basic and advanced settings \\ \hline				
				
				
			\end{tabular}
			\caption{\label{tab:settings_buttons} Functions of MatchMakr settings buttons}
		\end{table}	
	}
	
\buttontable


% ------------------------------------------------
%
% Figure
%
% ------------------------------------------------
\begin{figure}
	\centering
	\includegraphics[scale=\scalefactor]{settings.png}
	\caption{\label{fig:settings} MatchMakr Settings}
\end{figure}

% ------------------------------------------------
%
% Parameter Descriptions
%
% ------------------------------------------------
\begin{table}
	\centering
	\begin{tabular}[t]{| p{2in} | p{2.1in} | p{2.5in} |}
	
		% Header
		\tabheaderthree{Setting:}{Possible Values (min - max):}{Description:}
		
		% Questions
		\tablinethree{Path}{Any valid file path}
						{Folder where all of the files are stored.  This value can be changed by double-clicking this box, which will open a \texttt{File Open Dialog} box}	
		\tablinethree{Student Preferences}{Any valid \texttt{.csv} file name}
						{The name of the \texttt{.csv} file where the student preference data is
							stored.  By default, it is \texttt{student\_preferences.csv}.  This file should be found in the folder specified above in \texttt{Path}}	
		\tablinethree{Faculty Preferences}{Any valid \texttt{.csv} file name}
						{The name of the \texttt{.csv} file where the faculty preference data is
							stored.  By default, it is \texttt{faculty\_preferences.csv}.  This file should be found in the folder specified above in \texttt{Path}}
		\tablinethree{Student Availability}{Any valid \texttt{.csv} file name}
						{The name of the \texttt{.csv} file where the student availability data is
							stored.  By default, it is \texttt{student\_availability.csv}.  This file should be found in the folder specified above in \texttt{Path}.  If this file is not required, MatchMakr can be told not to look for it.  See Advanced Settings}
		\tablinethree{Faculty Availability}{Any valid \texttt{.csv} file name}
						{The name of the \texttt{.csv} file where the faculty availability data is
							stored.  By default, it is \texttt{faculty\_availability.csv}.  This file should be found in the folder specified above in \texttt{Path}.  If this file is not required, MatchMakr can be told not to look for it.  See Advanced Settings}	
		\tablinethree{Results Folder}{Any valid folder name}
						{The folder where the results will be output.  This is a subfolder of \texttt{Path} (specified above).}	
		\tablinethree{Student Schedules Folder Name}{Any valid folder name}
						{The folder where the student schedule \texttt{.pdf} files will be output.  This is a subfolder of \texttt{Results} (specified above)}	
		\tablinethree{Faculty Schedules Folder Name}{Any valid folder name}
						{The folder where the faculty schedule \texttt{.pdf} files will be output.  This is a subfolder of \texttt{Results} (specified above)}		
		\tablinethree{Number of Interview}{0 - 99}
						{The number of interview slots that are available.}		
		\tablinethree{Minimum Number of Interviews}{0 - (Number of Interviews)}
						{The minimum number of interview slots that someone can be scheduled for.  If someone is available for fewer interview slots than \texttt{Minimum Number of Interviews}, then the person will be assigned, at most, the number of interviews that they are available for.}		
		\tablinethree{Maximum Number of Interviews}
						{\pbthreelarge{(Minimum Number of Interviews) \\- (Number of Interviews)}}
						{The maximum number of interview slots that someone can be scheduled for}	
		\tablinethree{Recommend Extra Matches}{0 - 99}
						{Number of extra matches to recommend.  This helps people find good matches that could not be scheduled and explore those matches outside of scheduled time.}	
		\tablinethree{Faculty Advantage Factor}{0 - 100}
						{A value of 50 weights student and faculty preferences equally.  A value of 100 causes MatchMakr to only consider faculty preferences and not consider student preferences.  A value of 0 causes MatchMakr to only consider student preferences and not consider faculty preferences. Changing this parameter can improve matches.}					
		
		
	\end{tabular}
	\caption{\label{tab:settings} MatchMakr Settings}
\end{table}	



