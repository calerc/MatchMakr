\chapter{Processing Preference Data}

The Google Form puts most of the data in the correct format.  However, we must do a little bit of preprocessing before we feed the data into MatchMaker.  If we miss a step, it's okay.  MatchMakr can catch most errors and fix them automatically.

Example files can be found on the MatchMaker \href{\website}{website}

The following steps should be performed on both the faculty and student preference data:
\begin{enumerate}
	\item Open the preference data in any spreadsheet program (the most common is Microsoft Excel)
	\item Because the file is a \texttt{.csv}, you will have to import it when the import box pops up
		\begin{itemize}
			\item Select the following separators
				\begin{enumerate}
					\item comma (,)
					\item other, and then type in brackets([])
				\end{enumerate}
			\item Leave the rest of the settings as their defaults
		\end{itemize}
	\item To prevent data import in the future, save the preference files as \texttt{.xlsx}
		\begin{itemize}
			\item \textbf{ Whenever saving changes, first save the changes to the .xlsx file, then do \texttt{File} $\rightarrow$ \texttt{Save a copy} and save as a \texttt{.csv} file.}  MatchMakr cannot read \texttt{.xlsx} files.  It can only read \texttt{.csv} files.
		\end{itemize}
	\item If you did not use the expected MatchMakr question titles when making the survey, replace your titles with the expected MatchMakr titles
		\begin{itemize}
			\item Example: If you replaced ``Track'' with ``Common Interest'' when creating questions for the survey, now change ``Common Interest'' back to ``Track''
		\end{itemize}
	\item Insert a new column to the right of the ``Last Name'' column
	\item Title the new column ``Full Name''
	\item Concatenate the last names and first names:
		\begin{itemize}
			\item Assuming the last name data is in column C
			\item Assuming the first name data is in column D
			\item Assume that the preference data starts in row 2
			\item The formula for cell E2 should be: \texttt{=concatenate(C2, D2)}
			\item Drag this formula down to fill the rest of the column
		\end{itemize}
	\item To the bottom of the spreadsheet, add the names of anyone who did not fill out the survey
		\begin{itemize}
			\item If you don't have a list of people who didn't fill out the survey, MatchMakr can detect this in many circumstances.  See the chapter on Validating Settings and Running MatchMakr
			\item If you know any information about missing names, add it to the spreadsheet (e.g., faculty tracks)
		\end{itemize}
	\item Check the names for common mistakes and fix them
		\begin{itemize}
			\item Reversing first and last name
			\item Using a nickname or preferred name
			\item Using more names (hyphenated last names, middle name, etc...) then given in the original application
			\item MatchMakr can detect mistakes in many cases.  See the chapter on Validating Settings and Running MatchMakr
		\end{itemize}
	\item Alphabetize the list
		\begin{itemize}
			\item Go to \texttt{Data} $\rightarrow$ \texttt{Sort}
			\item Select \texttt{My data has column headers}
			\item Sort by:
				\begin{enumerate}
					\item Full Name
					\item Timestamp
				\end{enumerate}
		\end{itemize}
	\item Remove duplicate entries
	\item Remove all instances of names of people who are no longer attending
		\begin{itemize}
			\item Remove survey results (rows) from these people
			\item Remove the names of these people anywhere they appear in \textbf{BOTH} preference spreadsheets
			\item If the names appear in ranked lists, move lower-ranked entries up to fill the space
				\begin{itemize}
					\item As example, if a person who canceled was someone's ninth choice, replace the name of the cancel-ee with the tenth choice
				\end{itemize}
		\end{itemize}
	\item Save the preference files as \texttt{.xlsx} and also as \textt{.csv}
			
	
			
			
\end{enumerate}


\begin{figure}
	\centering
	\includegraphics[scale=\scalefactor]{example_preference_data.png}
	\caption{\label{fig:example_preference_data} Example of processed preference data}
\end{figure}

