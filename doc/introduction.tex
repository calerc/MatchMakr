

\chapter{Introduction}

\section{Purpose of MatchMakr}
MatchMakr is software that can be used to match interviewers with interviewees.  Originally, MatchMakr was designed to match prospective graduate students with faculty interviewers.  Because the words ``students'' and ``faculty'' are easier to differentiate than ``interviewer'' and ``interviewee,'' we will refer to interviewers as ``faculty'' or ``professors'' throughout this document, and we will refer to interviewees as ``students'' throughout this document.

\section{Why Use MatchMakr?}
\par
\textbf{MatchMakr is effective:} 
MatchMakr optimizes the preferences of everyone to ensure that both interviewers and interviewees are compatible.  MatchMakr can identify bad matches, and it avoids wasting interview time on those matches.  MatchMakr prioritizes mutual interest, and then fills in the gaps with the best matches available.

\par
\textbf{MatchMakr is efficient:}
Once data is collected, MatchMakr can make matches quickly and effortlessly while you do other things.

\par
\textbf{MatchMakr solves the "scheduling problem":}
MatchMakr removes the need for people to balance interview preferences with scheduling conflicts.  MatchMakr optimizes both at once.


\textbf{MatchMakr makes readable schedules:}
\begin{figure}[!h]
	\centering
	\includegraphics[scale=\scalefactor]{example_schedule.png}
	\caption{\label{fig:example_schedule} A schedule produced by MatchMakr}
\end{figure}

\section{Additional Resources}
Additional resources can be found at the MatchMakr \href{\website}{website}.  These resources include example surveys and a video version of this user manual.