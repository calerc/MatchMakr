\chapter{Defining Other Inputs}

In addition to the survey response data, it is necessary to create 1-3 other data files defining:
\begin{enumerate}
	\item Interview times
	\item Faculty availability
	\item Student availability
\end{enumerate}

It is always necessary to create an interview times file.  This file lets MatchMakr know when the interviews take place so that MatchMakr can print this information to the schedules.

If all people are available for all interview slots, it is not necessary to define the faculty availability and student availability files.  Frequently, faculty members have family commitments, so they are not available all day.  Students, however, tend to be available during all interview periods.  In this case, we can create a file defining faculty availability, but we don't need to create a file defining student availability.

Examples of files in the correct format can be found at the MatchMakr \href{https://sites.google.com/case.edu/matchmakr/home}{website}



% ------------------------------------------------
%
% Organizing files
%
% ------------------------------------------------
\section{Organizing your files}
At this point, you already have 2 files.  One is a \texttt{.csv} file that defines the preferences of students.  The other is a \texttt{.csv} file that defines the preferences of faculty.  You will be creating 1-3 new files.  It is important for MatchMakr that all of these files are in the same folder.  MatchMakr will also output all schedules to this directory.  For this reason, it is important that the data has a dedicated folder.  The location and name of the folder are not important so long as you know the location, and you have access to the folder.

From now on, we will refer to this folder as \texttt{MatchMakr\_Data\_Folder}

% ------------------------------------------------
%
% Prerequisites
%
% ------------------------------------------------
\section{Prerequisites}

In order to define the other inputs we need the following data:

\begin{enumerate}
	\item List of interview times
	\item List of students
	\item List of faculty
\end{enumerate}

% ------------------------------------------------
%
% Defining interview times
%
% ------------------------------------------------
\section{Defining interview times}

The following rules must be followed when defining the interview times file:
\begin{enumerate}
	\item It must be in the \texttt{.csv} format
	\item It must be in \texttt{MatchMakr\_Data\_Folder}
	\item It must be named \texttt{interview\_times.csv}
	\item It must contain data only in the first column
	\item Data must begin in the first row
	\item Data must be contiguous (don't skip cells)
	\item The number of interview times listed must match the number of interviews specified in MatchMakr \\
		\texttt{Settings} $\rightarrow$ \texttt{Number of Interviews} (see the Settings chapter)
	\item Interview times can be any string (text) of alphanumeric characters and symbols
\end{enumerate}

See Figure \ref{fig:valid_interview_times} for an example of a valid interview times file.

\begin{figure}
	\centering
	\includegraphics[scale=\scalefactorzoomin]{images/valid_interview_times.png}
	\caption{\label{fig:valid_interview_times} Example of a valid interview file.}
\end{figure}



% ------------------------------------------------
%
% Defining availability files
%
% ------------------------------------------------
\section{Defining availability files}

Frequently, members of the faculty have family commitments that prevent them from being available for interviews all day.  Because of this, we must specify faculty availability in a file.  Students are typically available during the entire interview period.  Thus, by default, MathMakr does not expect to receive student availability data.  MatchMakr can accept availability files from faculty, students, both, or neither.  This is specified in \texttt{Advanced Settings}.  See the chapter on Advanced Settings for more details.

The following rules must be followed when defining the availability files:
\begin{enumerate}
	\item It must be in the \texttt{.csv} format
	\item It must be in the \texttt{MatchMakr\_Data\_Folder}
	\item Separate files must be created for faculty and students
		\begin{itemize}
			\item By default, faculty availability must be specified
			\item By default, student availability does not need to be specified
			\item Settings can be configured in MatchMakr in \texttt{Advanced Settings}
		\end{itemize}
	\item Files may have any file name, however the file name must match the file name specified in MatchMakr \\ \texttt{Settings} $\rightarrow$ \texttt{faculty\_availability} and \texttt{student\_availability} 
		\begin{itemize}
			\item By default, faculty availability is specified in \texttt{faculty\_availability.csv}
			\item By default, student availability is specified in \texttt{student\_availability.csv}
		\end{itemize}
	\item A list of interview times should be placed in the $1^\text{st}$ column, starting at the $4^\text{th}$ row
		\begin{itemize}
			\item We recommend copying these interview times from \texttt{interview\_times.csv} 
		\end{itemize}
	\item A list of people (faculty or students, as appropriate) should be placed in the $2^\text{nd}$ row starting in the $2^\text{nd}$ column
		\begin{itemize}
			\item The list of people should be alphabetized
			\item The list of people must match the list of people listed in the preference files (\texttt{student\_preferences.csv} or \texttt{faculty\_preferences.csv}, as appropriate)
		\end{itemize}
	\item Count the number of people listed.  We will call this number $n$.
	\item Count the number of interview times.  We will call this number $i$
	\item Place the number 1 in all cells in columns 2-$n$ and rows 4-$i$
	\item Replace the number 1 with the number 0 in any cell corresponding to an interview time when a person is unavailable
\end{enumerate}

See Figure \ref{fig:valid_availability} for an example of a valid availability file.

\begin{figure}
	\centering
	\includegraphics[scale=\scalefactorzoomin]{images/valid_availability.png}
	\caption{\label{fig:valid_availability} Example of a valid availability file.}
\end{figure}

