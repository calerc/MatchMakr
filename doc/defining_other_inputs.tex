\chapter{Defining Other Inputs}

In addition to the survey response data, it is necessary to create 1-3 other data files defining:
\begin{enumerate}
	\item Interview Times
	\item Faculty availability
	\item Student Availability
\end{enumerate}

It is always necessary to create an interview times file.  This file lets MatchMakr know when the interviews take place so that MatchMakr can print this information to the schedules.

If all people are available for all interview slots, it is not necessary to define the faculty availability and student availability files.  Frequently, faculty members have family commitments, so they are not available all day.  Students, however, tend to be available during all interview periods.  In this case, we can create a file defining faculty availability, but we don't need to create a file defining student availability.



% ------------------------------------------------
%
% Organizing files
%
% ------------------------------------------------
\section{Organizing your files}
At this point, you already have 2 files.  One is a .csv file that defines the preferences of students.  The other is a .csv file that defines the preferences of faculty.  You will be creating 1-3 new files.  It is important for MatchMakr that all of these files are in the same folder.  MatchMakr will also output all schedules to this directory.  For this reason, it is important that the data has a dedicated folder.  The location and name of the folder are not important so long as you know the location, and you have access to the folder.

From now on, we will refer to this folder as \texttt{MatchMakr\_Data\_Folder}

% ------------------------------------------------
%
% Defining interview times
%
% ------------------------------------------------
\section{Defining interview times}



% ------------------------------------------------
%
% Defining availability files
%
% ------------------------------------------------
\section{Defining availability files}